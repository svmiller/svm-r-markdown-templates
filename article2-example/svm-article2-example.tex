% Options for packages loaded elsewhere
\PassOptionsToPackage{unicode}{hyperref}
\PassOptionsToPackage{hyphens}{url}
%
\documentclass[
  11pt,
]{article}
\usepackage{lmodern}
\usepackage{amssymb,amsmath}
\usepackage{ifxetex,ifluatex}
\ifnum 0\ifxetex 1\fi\ifluatex 1\fi=0 % if pdftex
  \usepackage[T1]{fontenc}
  \usepackage[utf8]{inputenc}
  \usepackage{textcomp} % provide euro and other symbols
\else % if luatex or xetex
  \usepackage{unicode-math}
  \defaultfontfeatures{Scale=MatchLowercase}
  \defaultfontfeatures[\rmfamily]{Ligatures=TeX,Scale=1}
  \setmainfont[]{cochineal}
  \setsansfont[]{Linux Biolinum O}
\fi
% Use upquote if available, for straight quotes in verbatim environments
\IfFileExists{upquote.sty}{\usepackage{upquote}}{}
\IfFileExists{microtype.sty}{% use microtype if available
  \usepackage[]{microtype}
  \UseMicrotypeSet[protrusion]{basicmath} % disable protrusion for tt fonts
}{}
\makeatletter
\@ifundefined{KOMAClassName}{% if non-KOMA class
  \IfFileExists{parskip.sty}{%
    \usepackage{parskip}
  }{% else
    \setlength{\parindent}{0pt}
    \setlength{\parskip}{6pt plus 2pt minus 1pt}
    }
}{% if KOMA class
  \KOMAoptions{parskip=half}}
\makeatother
\usepackage{xcolor}
\IfFileExists{xurl.sty}{\usepackage{xurl}}{} % add URL line breaks if available
% \IfFileExists{bookmark.sty}{\usepackage{bookmark}}{\usepackage{hyperref}}
% \hypersetup{
% %   pdftitle={Another Pandoc Markdown Article Starter and Template},
% % % % % %   pdfkeywords={pandoc, r markdown, knitr},
% % %   hidelinks,
% %   pdfcreator={LaTeX via pandoc}}
\urlstyle{same} % disable monospaced font for URLs
\usepackage[margin=1in]{geometry}
\usepackage{color}
\usepackage{fancyvrb}
\newcommand{\VerbBar}{|}
\newcommand{\VERB}{\Verb[commandchars=\\\{\}]}
\DefineVerbatimEnvironment{Highlighting}{Verbatim}{commandchars=\\\{\}}
% Add ',fontsize=\small' for more characters per line
\usepackage{framed}
\definecolor{shadecolor}{RGB}{248,248,248}
\newenvironment{Shaded}{\begin{snugshade}}{\end{snugshade}}
\newcommand{\AlertTok}[1]{\textcolor[rgb]{0.94,0.16,0.16}{#1}}
\newcommand{\AnnotationTok}[1]{\textcolor[rgb]{0.56,0.35,0.01}{\textbf{\textit{#1}}}}
\newcommand{\AttributeTok}[1]{\textcolor[rgb]{0.77,0.63,0.00}{#1}}
\newcommand{\BaseNTok}[1]{\textcolor[rgb]{0.00,0.00,0.81}{#1}}
\newcommand{\BuiltInTok}[1]{#1}
\newcommand{\CharTok}[1]{\textcolor[rgb]{0.31,0.60,0.02}{#1}}
\newcommand{\CommentTok}[1]{\textcolor[rgb]{0.56,0.35,0.01}{\textit{#1}}}
\newcommand{\CommentVarTok}[1]{\textcolor[rgb]{0.56,0.35,0.01}{\textbf{\textit{#1}}}}
\newcommand{\ConstantTok}[1]{\textcolor[rgb]{0.00,0.00,0.00}{#1}}
\newcommand{\ControlFlowTok}[1]{\textcolor[rgb]{0.13,0.29,0.53}{\textbf{#1}}}
\newcommand{\DataTypeTok}[1]{\textcolor[rgb]{0.13,0.29,0.53}{#1}}
\newcommand{\DecValTok}[1]{\textcolor[rgb]{0.00,0.00,0.81}{#1}}
\newcommand{\DocumentationTok}[1]{\textcolor[rgb]{0.56,0.35,0.01}{\textbf{\textit{#1}}}}
\newcommand{\ErrorTok}[1]{\textcolor[rgb]{0.64,0.00,0.00}{\textbf{#1}}}
\newcommand{\ExtensionTok}[1]{#1}
\newcommand{\FloatTok}[1]{\textcolor[rgb]{0.00,0.00,0.81}{#1}}
\newcommand{\FunctionTok}[1]{\textcolor[rgb]{0.00,0.00,0.00}{#1}}
\newcommand{\ImportTok}[1]{#1}
\newcommand{\InformationTok}[1]{\textcolor[rgb]{0.56,0.35,0.01}{\textbf{\textit{#1}}}}
\newcommand{\KeywordTok}[1]{\textcolor[rgb]{0.13,0.29,0.53}{\textbf{#1}}}
\newcommand{\NormalTok}[1]{#1}
\newcommand{\OperatorTok}[1]{\textcolor[rgb]{0.81,0.36,0.00}{\textbf{#1}}}
\newcommand{\OtherTok}[1]{\textcolor[rgb]{0.56,0.35,0.01}{#1}}
\newcommand{\PreprocessorTok}[1]{\textcolor[rgb]{0.56,0.35,0.01}{\textit{#1}}}
\newcommand{\RegionMarkerTok}[1]{#1}
\newcommand{\SpecialCharTok}[1]{\textcolor[rgb]{0.00,0.00,0.00}{#1}}
\newcommand{\SpecialStringTok}[1]{\textcolor[rgb]{0.31,0.60,0.02}{#1}}
\newcommand{\StringTok}[1]{\textcolor[rgb]{0.31,0.60,0.02}{#1}}
\newcommand{\VariableTok}[1]{\textcolor[rgb]{0.00,0.00,0.00}{#1}}
\newcommand{\VerbatimStringTok}[1]{\textcolor[rgb]{0.31,0.60,0.02}{#1}}
\newcommand{\WarningTok}[1]{\textcolor[rgb]{0.56,0.35,0.01}{\textbf{\textit{#1}}}}
\usepackage{graphicx}
\makeatletter
\def\maxwidth{\ifdim\Gin@nat@width>\linewidth\linewidth\else\Gin@nat@width\fi}
\def\maxheight{\ifdim\Gin@nat@height>\textheight\textheight\else\Gin@nat@height\fi}
\makeatother
% Scale images if necessary, so that they will not overflow the page
% margins by default, and it is still possible to overwrite the defaults
% using explicit options in \includegraphics[width, height, ...]{}
\setkeys{Gin}{width=\maxwidth,height=\maxheight,keepaspectratio}
% Set default figure placement to htbp
\makeatletter
\def\fps@figure{htbp}
\makeatother
\setlength{\emergencystretch}{3em} % prevent overfull lines
\providecommand{\tightlist}{%
  \setlength{\itemsep}{0pt}\setlength{\parskip}{0pt}}
\setcounter{secnumdepth}{-\maxdimen} % remove section numbering

\ifluatex
  \usepackage{selnolig}  % disable illegal ligatures
\fi
\usepackage[]{natbib}
\bibliographystyle{apsr}

\title{Another Pandoc Markdown Article Starter and Template\thanks{Replication files are available on the author's Github account
(\url{http://github.com/svmiller/svm-r-markdown-templates}).
\textbf{Current version}: September 14, 2020; \textbf{Corresponding
author}:
\href{mailto:steven.v.miller@gmail.com}{\nolinkurl{steven.v.miller@gmail.com}}.}}
\author{true \and true \and true}
\date{September 14, 2020}

% Jesus, okay, everything above this comment is default Pandoc LaTeX template. -----
% ----------------------------------------------------------------------------------
% I think I had assumed beamer and LaTex were somehow different templates.


\usepackage{kantlipsum}

\usepackage{abstract}
\renewcommand{\abstractname}{}    % clear the title
\renewcommand{\absnamepos}{empty} % originally center

\renewenvironment{abstract}
 {{%
    \setlength{\leftmargin}{0mm}
    \setlength{\rightmargin}{\leftmargin}%
  }%
  \relax}
 {\endlist}

\makeatletter
\def\@maketitle{%
  \newpage
%  \null
%  \vskip 2em%
%  \begin{center}%
  \let \footnote \thanks
      {\fontsize{14.5}{20}\selectfont\bfseries\sffamily\raggedright  \setlength{\parindent}{0pt} \@title \par}%
    }
%\fi
\makeatother


\title{Another Pandoc Markdown Article Starter and Template\thanks{Replication files are available on the author's Github account
(\url{http://github.com/svmiller/svm-r-markdown-templates}).
\textbf{Current version}: September 14, 2020; \textbf{Corresponding
author}:
\href{mailto:steven.v.miller@gmail.com}{\nolinkurl{steven.v.miller@gmail.com}}.}  }



%\author{\Large Steven V. Miller\vspace{0.05in} \newline\normalsize\emph{Clemson University}   \and \Large A Second Author Who Did Less Work\vspace{0.05in} \newline\normalsize\emph{The Ohio State University}   \and \Large A Graduate Student\vspace{0.05in} \newline\normalsize\emph{University of Alabama}  }


\date{}

\usepackage{titlesec}

% \sffamily\uppercase
\titleformat*{\section}{\large\bfseries\sffamily\uppercase}
\titleformat*{\subsection}{\bfseries\sffamily} % \small\uppercase
\titleformat*{\subsubsection}{\normalsize\itshape}
\titleformat*{\paragraph}{\normalsize\itshape}
\titleformat*{\subparagraph}{\normalsize\itshape}

% add some other packages ----------

% \usepackage{multicol}
% This should regulate where figures float
% See: https://tex.stackexchange.com/questions/2275/keeping-tables-figures-close-to-where-they-are-mentioned
\usepackage[section]{placeins}



\makeatletter
\@ifpackageloaded{hyperref}{}{%
\ifxetex
  \PassOptionsToPackage{hyphens}{url}\usepackage[setpagesize=false, % page size defined by xetex
              unicode=false, % unicode breaks when used with xetex
              xetex]{hyperref}
\else
  \PassOptionsToPackage{hyphens}{url}\usepackage[draft,unicode=true]{hyperref}
\fi
}

\@ifpackageloaded{color}{
    \PassOptionsToPackage{usenames,dvipsnames}{color}
}{%
    \usepackage[usenames,dvipsnames]{color}
}
\makeatother
\hypersetup{breaklinks=true,
            bookmarks=true,
            pdfauthor={Steven V. Miller (Clemson University) and A Second Author Who Did Less Work (The Ohio State University) and A Graduate Student (University of Alabama)},
             pdfkeywords = {pandoc, r markdown, knitr},  
            pdftitle={Another Pandoc Markdown Article Starter and Template},
            colorlinks=true,
            citecolor=blue,
            urlcolor=blue,
            linkcolor=magenta,
            pdfborder={0 0 0}}
\urlstyle{same}  % don't use monospace font for urls

% Add an option for endnotes. -----



% This will better treat References as a section when using natbib
% https://tex.stackexchange.com/questions/49962/bibliography-title-fontsize-problem-with-bibtex-and-the-natbib-package
\renewcommand\bibsection{%
   \section*{References}%
   \markboth{\MakeUppercase{\refname}}{\MakeUppercase{\refname}}%
  }%

% set default figure placement to htbp
\makeatletter
\def\fps@figure{htbp}
\makeatother



\usepackage{longtable}
\LTcapwidth=.95\textwidth
\linespread{1.05}
\usepackage{hyperref}
\begin{document}

% \textsf{\textbf{This is sans-serif bold text.}}
% \textbf{\textsf{This is bold sans-serif text.}}


% \maketitle

{% \usefont{T1}{pnc}{m}{n}
\setlength{\parindent}{0pt}
\thispagestyle{plain}
{%\fontsize{18}{20}\selectfont\raggedright
\maketitle  % title \par

}

%\textbf{\textsf{\Large Another Pandoc Markdown Article Starter and Template}}


{
   \vskip 13.5pt\relax \normalsize\fontsize{11}{12} 
   \MakeUppercase{\textsf{\large Steven V. Miller}}, \small{Clemson University}   \par \MakeUppercase{\textsf{\large A Second Author Who Did Less Work}}, \small{The Ohio State University}   \par \MakeUppercase{\textsf{\large A Graduate Student}}, \small{University of Alabama}   

}

}








\begin{abstract}

%    \hbox{\vrule height .2pt width 39.14pc}

    \vskip 8.5pt % \small 

\noindent \small{Lorem ipsum dolor sit amet, consectetur adipiscing elit. Donec sit amet
libero justo. Pellentesque eget nibh ex. Aliquam tincidunt egestas
lectus id ullamcorper. Proin tellus orci, posuere sed cursus at,
bibendum ac odio. Nam consequat non ante eget aliquam. Nulla facilisis
tincidunt elit. Nunc hendrerit pellentesque quam, eu imperdiet ipsum
porttitor ut. Interdum et malesuada fames ac ante ipsum primis in
faucibus. Suspendisse potenti. Duis vitae nibh mauris. Duis nec sem sit
amet ante dictum mattis. Suspendisse diam velit, maximus eget commodo
at, faucibus et nisi. Ut a pellentesque eros, sit amet suscipit eros.
Nunc tincidunt quis risus suscipit vestibulum. Quisque eu fringilla
massa.}


\vskip 8.5pt \noindent \emph{Keywords}: pandoc, r markdown, knitr \par

%    \hbox{\vrule height .2pt width 39.14pc}



\end{abstract}


\vskip -8.5pt


 % removetitleabstract


\setlength{\parindent}{16pt}
\setlength{\parskip}{0pt}

\hypertarget{introduction}{%
\section{Introduction}\label{introduction}}

\kant[1]

\hypertarget{a-subsection-in-the-introduction}{%
\subsection{A Subsection in the
Introduction}\label{a-subsection-in-the-introduction}}

\kant[2-5]

This is an R Markdown document. Markdown is a simple formatting syntax
for authoring HTML, PDF, and MS Word documents. For more details on
using R Markdown see \url{http://rmarkdown.rstudio.com}. Here's an
obligatory citation to \citet{xie2013ddrk}.

\hypertarget{literature-review}{%
\section{Literature Review}\label{literature-review}}

\kant[6-14]

\hypertarget{research-design}{%
\section{Research Design}\label{research-design}}

\kant[15]

\hypertarget{another-subsection-from-kant-who-writes-as-if-he-does-not-want-to-be-read}{%
\subsection{Another Subsection From Kant, Who Writes as If He Does Not
Want to Be
Read}\label{another-subsection-from-kant-who-writes-as-if-he-does-not-want-to-be-read}}

\kant[16-18]

\hypertarget{another-subsection}{%
\subsection{Another Subsection}\label{another-subsection}}

\kant[19-23]

\begin{figure}
\centering
\includegraphics{svm-article2-example_files/figure-latex/unnamed-chunk-1-1.pdf}
\caption{A Simple ggplot with the mtcars Data in R}
\end{figure}

When you click the \textbf{Knit} button a document will be generated
that includes both content as well as the output of any embedded R code
chunks within the document. You can embed an R code chunk like this:

\begin{Shaded}
\begin{Highlighting}[]
\KeywordTok{summary}\NormalTok{(cars)}
\end{Highlighting}
\end{Shaded}

\begin{verbatim}
##      speed           dist       
##  Min.   : 4.0   Min.   :  2.00  
##  1st Qu.:12.0   1st Qu.: 26.00  
##  Median :15.0   Median : 36.00  
##  Mean   :15.4   Mean   : 42.98  
##  3rd Qu.:19.0   3rd Qu.: 56.00  
##  Max.   :25.0   Max.   :120.00
\end{verbatim}

\hypertarget{results}{%
\section{Results}\label{results}}

\kant[24]

\hypertarget{including-plots}{%
\subsection{Including Plots}\label{including-plots}}

You can also embed plots, for example:

\includegraphics{svm-article2-example_files/figure-latex/pressure-1.pdf}

Note that the \texttt{echo\ =\ FALSE} parameter was added to the code
chunk to prevent printing of the R code that generated the plot.

\kant[25-35]

\begin{Shaded}
\begin{Highlighting}[]
\KeywordTok{plot}\NormalTok{(mtcars)}
\end{Highlighting}
\end{Shaded}

\includegraphics{svm-article2-example_files/figure-latex/unnamed-chunk-2-1.pdf}

\hypertarget{conclusion}{%
\section{Conclusion}\label{conclusion}}

\kant[36-40]

\newpage

  \bibliography{master.bib}

\end{document}

